\chapter{Background}
\label{background}

\section{Blockchain}

A blockchain is a decentralized, distributed ledger technology that enables secure, transparent, and tamper-resistant storage of digital records across a network of participants. It consists of a series of blocks, each containing a list of transactions, which are cryptographically linked and secured using cryptographic algorithms. This structure allows for enhanced security and data integrity, as altering the information in one block would require most network participants' consensus and modifying all subsequent blocks.

Although initially developed for supporting cryptocurrencies like Bitcoin, blockchain technology has evolved to accommodate various applications across many industries, such as finance, supply chain, healthcare, and more \cite{Xu2019} \cite{ibm-blockchain}. Ethereum has emerged as a leading platform for developing and deploying smart contracts among the different blockchain platforms.

\subsection{Etherum Blockchain}

The Ethereum blockchain, launched in 2015 by Vitalik Buterin and his team, was designed to facilitate the creation, management, and execution of decentralized applications (DApps) and smart contracts. Unlike Bitcoin, which is primarily used for transferring digital currency, Ethereum provides a decentralized virtual machine—the Ethereum Virtual Machine (EVM)—which can execute arbitrary Turing-complete code on the blockchain \cite{hildenbrandt2017kevm}. This feature allows developers to build and deploy more complex and versatile applications on the Ethereum platform.

Smart contracts are self-executing contracts with the terms of the agreement directly written into code. They automatically execute and enforce the contract's terms when predefined conditions are met without the need for intermediaries. This enables secure, decentralized, and automated transactions on the blockchain, leading to increased efficiency and reduced transaction costs. Ethereum's native cryptocurrency, Ether (ETH), is used to pay for the computational resources and transaction fees required to execute smart contracts on the network. A pertinent fact in this research is that once a smart contract is deployed on the Ethereum blockchain, it cannot be modified or removed. This immutability makes smart contracts a highly attractive target for attackers, as they can potentially cause significant financial losses and damage the organization's reputation. For this reason, smart contracts must be developed and deployed securely.

\subsection{The Solidity Language}

Solidity is a high-level, statically-typed, contract-oriented programming language specifically designed for writing smart contracts on the Ethereum blockchain. Created by Dr. Gavin Wood, Christian Reitwiessner, and their team at Ethereum, Solidity is influenced by other programming languages such as JavaScript, Python, and C++, and is designed to target the Ethereum Virtual Machine (EVM). The EVM executes the compiled bytecode of the smart contracts, which is generated from the Solidity source code.

\subsubsection{Syntax and Structure}

Solidity syntax is similar to JavaScript and employs a curly-bracket ({}) notation for defining code blocks. A Solidity smart contract typically starts with a \textit{pragma} directive, which specifies the version of the Solidity compiler required for the source code. This is followed by the contract definition, which includes the contract's state variables, functions, events, and access modifiers.

\begin{verbatim}
pragma solidity ^0.8.0;

contract SimpleStorage {
    uint256 private storedData;


    function set(uint256 x) public {
        storedData = x;
    }

    function get() public view returns (uint256) {
        return storedData;
    }

}
\end{verbatim}

The example above demonstrates a simple Solidity contract, \textit{SimpleStorage}, which allows users to store and retrieve an unsigned 256-bit integer value. The contract consists of a private state variable, \textit{storedData}, and two public functions, \textit{set()} and \textit{get()}.

\subsubsection{Data Types and Variables}

Solidity supports various data types, including value types (such as integers, booleans, and addresses) and reference types (such as arrays, mappings, and structs). Additionally, Solidity allows for the declaration of user-defined types, such as enums and structs, to create more complex data structures.

\subsubsection{Functions and Modifiers}

Functions in Solidity are similar to functions in other programming languages, defining a reusable block of code that performs a specific task. Functions can be declared as public, private, external, or internal, which determines their visibility and accessibility within the contract and by other contracts. Functions can also be marked as \textit{view} or \textit{pure}, indicating that they do not modify the contract's state and only read or compute data, respectively.

Modifiers can be used to alter the behavior of functions by appending or prepending additional code to the function's body. They are often used to enforce access control, by requiring certain conditions to be met before the function can be executed, such as requiring the sender to be the contract owner.

\subsubsection{Events and Inheritance}

Events are used in Solidity to emit logs that can be monitored by the contract's users, allowing them to be notified of specific occurrences or state changes within the contract. This is particularly useful for creating event-driven applications and tracking transactions on the Ethereum blockchain.

Solidity also supports inheritance, allowing contracts to inherit properties and methods from other contracts. This enables code reuse and modularity, facilitating the development of complex and robust smart contracts.

By understanding the fundamentals of Solidity and its features, developers can create secure and efficient smart contracts on the Ethereum platform. The background knowledge on Solidity provided in this subsection serves as a foundation for the subsequent analysis of smart contract vulnerabilities and the evaluation of ESBMC in this study.